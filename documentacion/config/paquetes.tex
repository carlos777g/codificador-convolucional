% -----------------------------------
% CONFIGURACIONES PERSONALIZADAS
% -----------------------------------

% Hoja A4 y márgenes
\usepackage[a4paper, margin=2.5cm]{geometry}

% Paquetes gráficos y visuales
\usepackage{graphicx}
\usepackage{float}
\usepackage{tikz,pgfplots}
\usepackage{xcolor}

% Codificación y tipografía
\usepackage[utf8]{inputenc}
\usepackage[T1]{fontenc}

% --- Tipo de letra ---
% \renewcommand{\familydefault}{\sfdefault} % Arial/Helvetica
% Tipo de letra serif (como Times New Roman)
\usepackage{newtxtext}  % Times
% Si quieres forzar helvetica más directamente:
% \usepackage{helvet}
% \renewcommand{\familydefault}{\sfdefault}

% Código fuente
\usepackage{listings}
\lstset{
  language=C++,
  basicstyle=\ttfamily\small,
  keywordstyle=\color{blue},
  stringstyle=\color{red},
  commentstyle=\color{green!50!black},
  morekeywords={lambda},
  breaklines=true,
  showstringspaces=false,
  numbers=left,
  numberstyle=\tiny\color{gray},
  frame=single,
  captionpos=b
}


% Bibliografía con biblatex
\usepackage[backend=bibtex, style=numeric]{biblatex}
\addbibresource{bib/referencias.bib}

% Hipervínculos clickeables
\usepackage[
  colorlinks=true,
  linkcolor=black,
  citecolor=black,
  urlcolor=blue
]{hyperref}

% Estilos para títulos
\usepackage{titlesec}
\titleformat{\section}
  {\color{cyan!45!black}\normalfont\Large\bfseries}{\thesection.}{1em}{}

\titleformat{\subsection}
  {\color{black}\normalfont\large\bfseries}{\thesubsection.}{1em}{}
  
% Índice personalizado (si usas tabla de contenido)
\usepackage{tocloft}
\renewcommand{\cftsecaftersnum}{.}
\renewcommand{\cftsubsecaftersnum}{.}

% Secciones con numeración romana
\renewcommand{\thesection}{\Roman{section}}
\renewcommand{\thesubsection}{\thesection.\Alph{subsection}}
